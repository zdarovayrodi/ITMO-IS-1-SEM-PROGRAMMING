\documentclass[a4paper,14pt]{article}
\usepackage[utf8]{inputenc}
\usepackage[english, russian]{babel}
\usepackage{amsmath}
\usepackage{geometry} % Меняем поля страницы
\geometry{left=2cm}% левое поле
\geometry{right=2cm}% правое поле
\geometry{top=2cm}% верхнее поле
\geometry{bottom=2cm}% нижнее поле
\usepackage[utf8]{inputenc}
\usepackage{amssymb,amsfonts,amsmath,cite,enumerate,float,indentfirst}

\usepackage[dvips]{graphicx}
\usepackage{multirow}
\graphicspath{{images/}}

% Default fixed font does not support bold face
\DeclareFixedFont{\ttb}{T1}{txtt}{bx}{n}{10} % for bold
\DeclareFixedFont{\ttm}{T1}{txtt}{m}{n}{10}  % for normal

% Custom colors
\usepackage{color}

\definecolor{deepblue}{rgb}{0,0,0.5}
\definecolor{deeporange}{rgb}{1,0.5,0.5}
\definecolor{deepred}{rgb}{0.8,0,0}
\definecolor{deepgreen}{rgb}{0,0.5,0}
\definecolor{grey1}{rgb}{0.5,0.5,0.5}
\usepackage{listings}

% Python style for highlighting
\newcommand\pythonstyle{\lstset{
        language=Python,
        basicstyle=\small\ttm,
        otherkeywords={self},             % Add keywords here
        keywordstyle=\small\ttb\color{deeporange},
        emph={},          % Custom highlighting
        emphstyle=\small\color{deepred},    % Custom highlighting style
        stringstyle=\small\color{deepgreen},
        commentstyle=\small\color{grey1}\ttm,
        % frame=tb,                         % Any extra options here
        showstringspaces=false,           %
        breaklines=true,
        tabsize=2
}}
% Python environment
\lstnewenvironment{python}[1][] {
    \pythonstyle
    \lstset{#1}
}{}
% use: \begin{python}...\end{python}

% Python for inline
\newcommand\pythoninline[1]{{\pythonstyle\lstinline!#1!}}
% use: \pythoninline{...}

\author{Шайдулин Михаил}
\date{M3106}
\title{\textbf{Лабораторная работа №1}}

\begin{document}

\begin{titlepage}
\newpage

\begin{center}
Министерство науки и высшего образования Российской Федерации\\
% \vspace{3em}
Федеральное государственное автономное образовательное учреждение высшего образования\\
«Национальный исследовательский университет ИТМО»\\
Факультет информационных технологий и программирования\\
\end{center}

\vspace{\fill}
% \vspace{28em}

\begin{center}
Программирование\\
\textbf{Лабораторная работа №4}\\
\end{center}

\vspace{\fill}
% \vspace{20em}

\newbox{\lbox}
\savebox{\lbox}{\hbox{Шайдулин Михаил Андреевич}}
\newlength{\maxl}
\setlength{\maxl}{\wd\lbox}
\hfill\parbox{14cm}{
\hspace*{5cm}Выполнил студент:\hfill\hbox to\maxl{Шайдулин Михаил Андреевич \hfill}\\
\hspace*{5cm}Группа:\hfill\hbox to\maxl{M3106}\\
}


\vspace{8em}
% \vspace{\fill}

\begin{center}
Санкт-Петербург \\2021
\end{center}

\end{titlepage}


% 1
\section{metaeditor.c}
\begin{python}
#include <stdio.h>
#include <string.h>
#include <stdlib.h>
#include <locale.h>



int main(int argc, char **argv)
{
    char* songname;
    char* prop_name;
    char* prop_value;
    uint8_t show;
    int get_flag = 0;
    int show_flag = 0;

    // for loop to initialize the parameters from terminal
    for (int argv_index = 1; argv_index < argc; argv_index++)
    {
        // file path parameter
        if (argv[argv_index][2] == 'f')
        {
            // path to a song
            songname = argv[argv_index] + 11;
            // code here
        }
        // show parameter
        else if (argv[argv_index][2] == 's')
        {
            show_flag = 1;
        }
        // get parameter
        else if (argv[argv_index][2] == 'g')
        {
            prop_name = argv[argv_index] + 6;
            get_flag = 1;
        }
        // set parameter
        else if (argv[argv_index][2] == 's')
        {
            prop_name = argv[argv_index] + 6;
            // code here
        }
        else if (argv[argv_index][2] == 'v')
        {
            prop_value = argv[argv_index] + 8;
            // code here
        }
    }

    // opening file.mp3
    FILE * music_file = fopen(songname, "rb+");
    if (music_file == NULL)
    {
        printf ("Error opening %s\n", songname);
        exit(1);
    }
    // initializing file size
    fseek (music_file , 0 , SEEK_END);  // setting position to the end of the file
    long lSize = ftell (music_file);    // getting size in bytes
    fseek (music_file , 0 , SEEK_SET); // setting pointer to the start of the file
    printf("%ld bytes\n", lSize); //

    char signature[4];
    char name_of_frame[5];
    char artist[30];
    char album[30];
    char year[4];
    char description[30];
    unsigned short ganre;

    fread(signature, 1, sizeof(signature)-1, music_file); // reading signature
    signature[3] = '\0'; // closing line

    fseek(music_file, 3, SEEK_CUR);
    int size;
    char bsize[4];
    fread(bsize, sizeof(bsize), 1, music_file);
    size = ((bsize[0] << 21) | (bsize[1] << 14) | (bsize[2] << 7) | (bsize[3]));
    printf("signature: %s\n\n",signature);

    while (ftell(music_file) < size) {
        fread(name_of_frame, 1, sizeof(name_of_frame)-1, music_file); // reading in buffer
        name_of_frame[4] = '\0'; // closing line

        int size_of_frame;
        char fsize[4];
        fread(fsize, 1, sizeof(fsize), music_file); // reading
        size_of_frame = ((fsize[0] << 24) | (fsize[1] << 16) | (fsize[2] << 8) | (fsize[3]));
        fseek(music_file, 2, SEEK_CUR);
        int unicode = 0;
        unicode = fgetc(music_file);
        char * frame_content = (char *)malloc(size_of_frame);
        fread(frame_content, sizeof(char), size_of_frame-1, music_file);
        frame_content[size_of_frame-1] = '\0'; // closing line

        if (name_of_frame && !name_of_frame[0])
        {
            break;
        }
        if (show_flag == 1)
        {
            //printing content
            printf("frame: %s, %d bytes\n", name_of_frame, size);
            printf("content: %s\n\n", frame_content);
        } else if (get_flag == 1 && strcmp(name_of_frame, prop_name) == 0)
        {
            printf("frame: %s, %d bytes\n", name_of_frame, size);
            printf("content: %s\n\n", frame_content);
        }
    }

    // } else {
        // printf("frame_content: %s\n", frame_content);
    // }
    // shutting down
    fclose (music_file);
    return 0;
}
\end{python}

\end{document}
