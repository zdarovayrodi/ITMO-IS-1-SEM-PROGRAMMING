\documentclass[a4paper,14pt]{article}
\usepackage[utf8]{inputenc}
\usepackage[english, russian]{babel}
\usepackage{amsmath}
\usepackage{geometry} % Меняем поля страницы
\geometry{left=2cm}% левое поле
\geometry{right=2cm}% правое поле
\geometry{top=2cm}% верхнее поле
\geometry{bottom=2cm}% нижнее поле
\usepackage[utf8]{inputenc}
\usepackage{amssymb,amsfonts,amsmath,cite,enumerate,float,indentfirst}

\usepackage[dvips]{graphicx}
\usepackage{multirow}
\graphicspath{{images/}}

% Default fixed font does not support bold face
\DeclareFixedFont{\ttb}{T1}{txtt}{bx}{n}{10} % for bold
\DeclareFixedFont{\ttm}{T1}{txtt}{m}{n}{10}  % for normal

% Custom colors
\usepackage{color}

\definecolor{deepblue}{rgb}{0,0,0.5}
\definecolor{deeporange}{rgb}{1,0.5,0.5}
\definecolor{deepred}{rgb}{0.8,0,0}
\definecolor{deepgreen}{rgb}{0,0.5,0}
\definecolor{grey1}{rgb}{0.5,0.5,0.5}
\usepackage{listings}

% Python style for highlighting
\newcommand\pythonstyle{\lstset{
        language=Python,
        basicstyle=\small\ttm,
        otherkeywords={self},             % Add keywords here
        keywordstyle=\small\ttb\color{deeporange},
        emph={},          % Custom highlighting
        emphstyle=\small\color{deepred},    % Custom highlighting style
        stringstyle=\small\color{deepgreen},
        commentstyle=\small\color{grey1}\ttm,
        % frame=tb,                         % Any extra options here
        showstringspaces=false,           %
        breaklines=true,
        tabsize=2
}}
% Python environment
\lstnewenvironment{python}[1][] {
    \pythonstyle
    \lstset{#1}
}{}
% use: \begin{python}...\end{python}

% Python for inline
\newcommand\pythoninline[1]{{\pythonstyle\lstinline!#1!}}
% use: \pythoninline{...}

\author{Шайдулин Михаил}
\date{M3106}
\title{\textbf{Лабораторная работа №1}}

\begin{document}

\begin{titlepage}
\newpage

\begin{center}
Министерство науки и высшего образования Российской Федерации\\
% \vspace{3em}
Федеральное государственное автономное образовательное учреждение высшего образования\\
«Национальный исследовательский университет ИТМО»\\
Факультет информационных технологий и программирования\\
\end{center}

\vspace{\fill}
% \vspace{28em}

\begin{center}
Программирование\\
\textbf{Лабораторная работа №3}\\
\end{center}

\vspace{\fill}
% \vspace{20em}

\newbox{\lbox}
\savebox{\lbox}{\hbox{Шайдулин Михаил Андреевич}}
\newlength{\maxl}
\setlength{\maxl}{\wd\lbox}
\hfill\parbox{14cm}{
\hspace*{5cm}Выполнил студент:\hfill\hbox to\maxl{Шайдулин Михаил Андреевич \hfill}\\
\hspace*{5cm}Группа:\hfill\hbox to\maxl{M3106}\\
}


\vspace{8em}
% \vspace{\fill}

\begin{center}
Санкт-Петербург \\2021
\end{center}

\end{titlepage}

\section{lab3.c}
\begin{python}
#include <stdio.h>
#include <stdlib.h>
#include <string.h>

int main(int argc, char **argv) {
    int i;
    int n=128, statusError5xxCounter=0;
    char * string;
    string = (char*)malloc(n*sizeof(char));
    char c;
    FILE * file=fopen("NASA_access_log_Jul95", "r");
    FILE * fout=fopen("result.out", "w");

    while((c = fgetc(file)) != EOF)
    {
        i = 0;
           while ((c!='\n') && (c != EOF))
           {
               if(i>n-1)
            {
                n+=128;
                string = (char*)realloc(string, n*sizeof(char));
            }

            string[i] = c;
            i++;
            c = fgetc(file);
        }
        int requestFirstIndex = 0, requestSecondIndex = 0, index = 0;
        while (requestSecondIndex == 0)
        {
            if ((string[index] == '"') && (requestFirstIndex == 0))
            {
                requestFirstIndex = index+1;
            } else if ((string[index] == '"') && (requestSecondIndex == 0))
            {
                requestSecondIndex = index+1;
            }
            index++;
        }
        if (string[requestSecondIndex+1] == '5')
        {
            statusError5xxCounter++;
            for (int j=requestFirstIndex; j<requestSecondIndex-1; j++)
                fprintf(fout, "%c", string[j]);
            fprintf(fout, "\n");
        }
    }
    free(string);
    fprintf(fout, "Number of request resulting in 5xx response: %d\n", statusError5xxCounter);
    fclose(fout);
    fclose(file);
    return 0;
}
\end{python}

\end{document}
