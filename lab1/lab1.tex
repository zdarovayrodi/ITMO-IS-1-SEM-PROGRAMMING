\documentclass[a4paper,14pt]{article}
\usepackage[utf8]{inputenc}
\usepackage[english, russian]{babel}
\usepackage{amsmath}
\usepackage{geometry} % Меняем поля страницы
\geometry{left=2cm}% левое поле
\geometry{right=2cm}% правое поле
\geometry{top=2cm}% верхнее поле
\geometry{bottom=2cm}% нижнее поле
\usepackage[utf8]{inputenc}
\usepackage{amssymb,amsfonts,amsmath,cite,enumerate,float,indentfirst}

\usepackage[dvips]{graphicx}
\usepackage{multirow}
\graphicspath{{images/}}

% Default fixed font does not support bold face
\DeclareFixedFont{\ttb}{T1}{txtt}{bx}{n}{10} % for bold
\DeclareFixedFont{\ttm}{T1}{txtt}{m}{n}{10}  % for normal

% Custom colors
\usepackage{color}

\definecolor{deepblue}{rgb}{0,0,0.5}
\definecolor{deeporange}{rgb}{1,0.5,0.5}
\definecolor{deepred}{rgb}{0.8,0,0}
\definecolor{deepgreen}{rgb}{0,0.5,0}
\definecolor{grey1}{rgb}{0.5,0.5,0.5}
\usepackage{listings}

% Python style for highlighting
\newcommand\pythonstyle{\lstset{
        language=Python,
        basicstyle=\small\ttm,
        otherkeywords={self},             % Add keywords here
        keywordstyle=\small\ttb\color{deeporange},
        emph={},          % Custom highlighting
        emphstyle=\small\color{deepred},    % Custom highlighting style
        stringstyle=\small\color{deepgreen},
        commentstyle=\small\color{grey1}\ttm,
        % frame=tb,                         % Any extra options here
        showstringspaces=false,           %
        breaklines=true,
        tabsize=2
}}
% Python environment
\lstnewenvironment{python}[1][] {
    \pythonstyle
    \lstset{#1}
}{}
% use: \begin{python}...\end{python}

% Python for inline
\newcommand\pythoninline[1]{{\pythonstyle\lstinline!#1!}}
% use: \pythoninline{...}

\author{Шайдулин Михаил}
\date{M3106}
\title{\textbf{Лабораторная работа №1}}

\begin{document}

\begin{titlepage}
\newpage

\begin{center}
Министерство науки и высшего образования Российской Федерации\\
% \vspace{3em}
Федеральное государственное автономное образовательное учреждение высшего образования\\
«Национальный исследовательский университет ИТМО»\\
Факультет информационных технологий и программирования\\
\end{center}

\vspace{\fill}
% \vspace{28em}

\begin{center}
Программирование\\
\textbf{Лабораторная работа №1}\\
\end{center}

\vspace{\fill}
% \vspace{20em}

\newbox{\lbox}
\savebox{\lbox}{\hbox{Шайдулин Михаил Андреевич}}
\newlength{\maxl}
\setlength{\maxl}{\wd\lbox}
\hfill\parbox{14cm}{
\hspace*{5cm}Выполнил студент:\hfill\hbox to\maxl{Шайдулин Михаил Андреевич \hfill}\\
\hspace*{5cm}Группа:\hfill\hbox to\maxl{M3106}\\
}


\vspace{8em}
% \vspace{\fill}

\begin{center}
Санкт-Петербург \\2021
\end{center}

\end{titlepage}


% 1
\section{Текст программы}
\begin{python}
#include <stdio.h>
#include <string.h>
#include <ctype.h>
#define IN 1 /* внутри слова */
#define OUT 2 /* вне слова */

int main(int argc, char **argv) {
	FILE *file = fopen(argv[1], "r");

	/* открыть не удалось */
    if (file == NULL) {
       printf("Не удалось открыть файл \"%s\".\n", argv[1]);
       return 0;
    }

	long long counter = 0;
	int str;
	int flag = 0;

	/* посчитать строки */
	if (strcmp(argv[2], "-l") == 0 || strcmp(argv[2], "--lines") == 0) {
		while ((str = (fgetc(file))) != EOF) {
			if (flag == 0) {
				counter++;
				flag = 1;
			} else if (str == '\n') {
				flag = 0;
			}
		}
      printf("Lines: ");
	}

	/* посчитать слова */
	if (strcmp(argv[2], "-w") == 0 || strcmp(argv[2], "--words") == 0) {
		unsigned short int status = OUT;
		while ((str = (fgetc(file))) != EOF) {
    		if (status == OUT && str != ' ' && str != '\n' && str != '\t') {
    			counter++;
    			status = IN;
    		}
    		else if (str == ' ' || str == '\n' || str == '\t') {
    			status = OUT;
        	}
      }
      printf("Words: ");
	}

	/* посчитать байты (все символы) */
	if (strcmp(argv[2], "-c") == 0 || strcmp(argv[2], "--bytes") == 0) {
      while ((str = (fgetc(file))) != EOF) {
          counter++;
      }
	printf("Bytes: ");
	}


	printf("%lld\n", counter);
	return 0;
}

\end{python}

\end{document}
